\documentclass[a4paper,12pt]{article}
%Note that this is not a perfect essay as the prompt demands. Details in parenthesis may be omitted and our conclusions should be restated in the last paragraph. Also, you may consider simplifying some of the discussions on par with the word count.
\begin{document}
The increasing amount of Physics knowledge transformed into everyday life technology marks the inevitable raise in the volume of technology perceived by the public. The rising number of students walking with their heads down and eyes fixed on their smartphones on a university campus is perhaps the most conspicuous case for educators. This passage aims to briefly discuss smart phone's impact on students.

As with any other rigorous discussions, we would first like to define the term \textit{smart phone}. Smartphone in this passage refers to any handheld device that functions as a computer. From this definition, we can infer two of its principal impacts on students. The handhold property means that such device is more readily accessible than traditional computers like laptops. Such ease in accessibility gives rise to the growing number of time and occasions in which students are tempted to use smartphones (P1). P1 coupled with the computer function of smartphones brings an increase in the amount of information students consume (P2). Further, from P1, we can also deduce the fact that smartphones hurt students' eyesight.
%To comply with the word limit, we sadly have to wrap up our discussion here. (it should be obvious to any reader with a normal degree of sanity that such a bloody short essay is by no means a complete discussion of the impact of  smartphones on campus)

From the impacts presented above, I suggest placing a hard limit (with penalties associated with students' GPA) on the amount of time students are allowed to use smartphones on campus. Also to force the installation of apps for educational purposes.
\end{document}
