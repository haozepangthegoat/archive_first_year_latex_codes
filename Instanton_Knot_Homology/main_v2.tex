\documentclass{amsart}
\usepackage{amsmath}
\usepackage{geometry}
\usepackage{amsthm}
\usepackage{fancyhdr}
\usepackage{color}


\title{The calculation of dimensions of Framed instanton Floer homology of $3$-dimension flow with Knot K from Ball $3$D after Dehn Surgery}
\author{Yu-Chia Jen,Haoze Pang}
\address{International Division, Guangzhou Zhixin High School}
\curraddr{12 Shuiyin Erheng Road, Yuexiu District, Guangzhou, Guangdong, China}
\email{692578536@qq.com}
\newtheorem{thm}{Theorem}[section]
\theoremstyle{definition}
\newtheorem{defn}[thm]{Definition}
\newtheorem{cor}[thm]{Corollary}
\newtheorem{prop}[thm]{Proposition}
\newcommand{\im}{{\rm im~}}
\date{November 2021}


\begin{document}
\begin{abstract}
\end{abstract}
\maketitle
\tableofcontents


\section{Introduction}
Knots theory is an important branch of topology which originated in 20th century. It studies embedded circle in $3$-dimensional manifolds. It has many important applications in theoretical physics, molecular chemistry and composition of biological molecule such as proteins and DNA and so on.

In this paper, we focus on knots inside $\mathbb{R}^3$ and $S^3$. Especially we study the Dehn surgery on knots and a special type of homological invariants of the Dehn surgery called the framed instanton Floer homology, which is denoted by $I^{\sharp}$. A Dehn surgery of a knot $K\subset S^3$ is to remove a tubular neighborhood $N(K)$ of $K$ from $S^3$, and then re-glue it back using a different identification of the toroidal boundary. Note any connected non-separating simple closed curve on a torus can be described as $p$ copies of $m_K$ and $q$ copies of $l_K$, where $m_K$ is the meridian of the knot, $l_K$ is the Seifert longitude of the knot, and $p, q$ are co-prime integers. Different ways of gluing $N(K)$ back give rise to different $3$-manifolds. It turns out that the gluing $3$-manifold only depends on the image of $m_K$.

Suppose $m_K$ is glued to $pm_c+ql_c$, where $m_c$ is the compliment of $m_k$ and $l_c$ is the compliment of $l_K$. This give rise to a new 3-manifold $S^3_{p/q}(K)$. This new space is called the Dehn surgery along $K$ of slope $p/q$. We have special interest when $p/q=2g(K)+1$, where $g(K)$ is the genus of knot $k$.

The objective of the paper is to estimate the total dimension of  $I^{\sharp}(S^{3}_{5}(K))$ of Dehn surgery along knot $7_6$. The rest of the paper is organized as follow: Section \ref{sec2} introduces some necessary definitions of knots theory and linear algebra. In section \ref{sec3}, we calculate the estimated dimension of Knot $7_6$ and show the calculating process.   

\section{Preliminaries}\label{sec2}
\begin{defn}
{\bf Knot}
\end{defn}
A {\bf knot} $K$ is an embedded circle inside the $3$-dimensional Euclidean space $\mathbb{R}^3$:
\[K:S^1\hookrightarrow\mathbb{R}^3.\]
Knots are defined inside the $3$-dimensional space. Because it is hard  to describe and study objects in dimension $3$, we introduce the following definition for our convenience.

\begin{defn}
A {\bf knot diagram} is a projection of the knot onto some plane in $\mathbb{R}^3$ which only allows double self-intersection.
\end{defn}
%A knot diagram is a projection of the knot onto some plane in $\mathbb{R}^3$ which only allow double intersection. There are infinitely many different embeddings of circles into $\mathbb{R}^3$, but many of them can have identical properties. So in order to study different kinds of knots, we make the following definition.
There are infinitely many different embeddings of circles into $\mathbb{R}^3$, but many of them share similar properties. So in order to study different kinds of knots, we make the following definition.

\begin{defn}
For a knot $K$, if it has a diagram so that when we transverse the knot, the over and under crossings appear alternatively, then we call this knot an {\bf alternating} knot.
\end{defn}


\begin{defn}
{\bf Kernel of $f$}
\end{defn}
Suppose $V$ and $W$ are two vectors spaces,$f:V\rightarrow W$ is a linear map. In linear algebra, the {\bf kernel} of a linear map is a set of $v$ for which $v \in V$ satisfies $f(v)=0$, $0$ represents the zero vector in $W$:
\[\ker (f)= \{v\in V\mid f(v)=\vec{0} \in W\}.\]

\begin{defn}
{\bf Image of $f$}
\end{defn}
Suppose $V$ and $W$ are two vectors spaces,$f:V\rightarrow W$ is a linear map. In linear algebra, the {\bf image} of a linear map is a set of $w$ for which $w \in W$ satisfies $f(v)=w,\exists v \in V$:
\[\im (f)= \{w\in W \mid\exists v\in V,f(v)= w\}.\]

\begin{thm}[{\bf Fundamental theorem of linear algebra}]
Suppose $f:U\rightarrow V$ is a linear map between finite dimensional vector spaces, then:
\[\dim \ker (f)+\dim \im (f)=\dim U.\]
\end{thm}

\begin{thm} \label{thm: A}
Supposed $A, B$ are both subspaces of $C$, then we have:
\[\max\{0,\dim A+\dim B-\dim C\}\leq \dim(A\cap B)\leq \min\{\dim A, \dim B\}.\]
\end{thm}

\begin{thm}[{\cite[Chapter 7 and 8]{rolfsen2003knot}}]\label{thm: KM}
For any knot $K\subset \mathbb{R}^3$, there exists a polynomial associated to it, which is called {\bf the Alexander polynomial}, and is denoted by
\begin{equation}\label{eq: alexander polynomial}
	\Delta_K(t)=a_{n}t^{n}+a_{n-1}t^{n-1}+...+a_{0}+...+a_{-(n-1)}t^{-(n-1)}+a_{-n}t^{-n}.
\end{equation}
{\bf The Alexander polynomials} of knots satisfy the following properties:
\begin{enumerate}
\item If $K$ is the unknot, then $\Delta_K(1)=1.$
\item
For any $i\in\mathbb{Z}$, we have $a_{i}=a_{-i}.$
\end{enumerate}
\end{thm}

\begin{thm}[Kronheimer and Mrowka \cite{kronheimer2010knots}]\label{thm: KM}
For any knot $K \subset\mathbb{R}^3$, there is a finite dimensional vector space, denoted by $KHI(K)$, associated to $K$. It is called {\bf the instanton knot homology} of $K$. It satisfies the following properties:
\begin{enumerate}
\item
There is a decomposition:
\[KHI(K)=\bigoplus_{i\in\mathbb{Z}} KHI(K,i).\]
Here $\bigoplus$ means the direct sum of vector spaces. The vector space $KHI(K,i)$ is called the {\bf $i$-th graded part} of $KHI(K)$.
\item
Suppose $\Delta_K(t)$ is the Alexander polynomial of $K$ and $a_{i}'s$ are the coefficients of $\Delta_K$ as in (\ref{eq: alexander polynomial}). Then we have
\[\dim KHI(K,i)\geq |a_{i}|\]
for all $i\in\mathbb{Z}$. Furthermore, the integer ($\dim KHI(K,i)-|a_{i}|$) is divisible by 2 for all $i\in\mathbb{Z}$.
\item For any $i\in\mathbb{Z}$, we have
\[KHI(K,i)\cong KHI(K,-i).\]
\item
If $K$ is an alternating knot, then we have
\[\dim KHI(K,i)=|a_{i}|\]
for all $i\in\mathbb{Z}$.
\end{enumerate}
\end{thm}

\begin{defn}
Suppose we have a sequence of vector spaces $C_{1},C_{2}$,... Suppose for each $i\in\mathbb{Z}$, we have a linear map
\[d_{i}:C_{i}\rightarrow C_{i-1}.\]
$d$ is called the {\bf differential}. The couple $({C_{i},{d_{i}}})$ is called a {\bf chain complex} if the composition of maps
\[d_{i}\circ d_{i+1}:C_{i}\rightarrow C_{i-1}\]
is always zero for all $i\in\mathbb{Z}$. Let
\[C=\bigoplus_{i\in\mathbb{Z}} C_{i}\]
and let d be the linear map
\[d:C\rightarrow C\]
so that d restricts to $d_{i}$ on $C_{i}$, i.e.,
\[d|_{C_{i}}=d_{i}:C_{i}\rightarrow C_{i-1}.\]
Then we usually write $(C,d)$ instead of $({C_{i}},{d_{i}})$.
\end{defn}

\begin{defn}
Suppose $(C,d)$ is a chain complex. Define
\[(H,d)=\ker (d)/\im (d)\]
to be the {\bf homology} of $(C,d)$. Here $\ker$($d$) means the kernel of $d$ and $\im$($d$) means the image of $d$.
The symbol/means the {\bf quotient} of vector spaces. If
\[C=\bigoplus_{i\in\mathbb{Z}} C_{i}\]
and let $d$ be the linear map
\[d:C\rightarrow C\]
so that $d$ restricts to $d_{i}$ on $C_{i}$, i.e.,
\[d|_{C_{i}}=d_{i}:C_{i}\rightarrow C_{i-1}.\]
Then we can define
\[H_{i}(C,d)=\ker (d_{i})/\im (d_{i+1})\]
and we have:
\[H(C,d)=\bigoplus_{i\in\mathbb{Z}} H_{i}(C,d).\]
\end{defn}

\begin{thm} [Li and Ye \cite{LY2020}]\label{thm: Li-Ye}
\end{thm}
Suppose $K\subset \mathbb{R}^3$ is a knot, and $KHI(K)$ is its instanton knot homology. Then the following is true.
\begin{enumerate}
\item
For any $i,j\in \{-g,...,g\}$, where $g$ is the genus of $K$, there exist a map:
\[d^{i}_{j}:KHI(k,i)\rightarrow KHI(k,j)\]
such that if $k+j=2i$, then $d^{i}_{j}\circ d^{k}_{i}=0$.
\item
For any $s\in\mathbb{Z}$, define
\[C_s=KHI(K){\rm~and~}d_s=\sum_{i>j\geq s}d^j_i+\sum_{i<j\leq s}d^j_i.\]
Then the following is true.
\begin{itemize}
\item [a)] If $s=g$ or $-g$, we have
\[\dim H(C_s,d_s)=1.\]
\item [b)] We have an isomorphism
\begin{equation}\label{eq: bent complex}
I^{\sharp}(S^3_{2g+1}(K))\cong\bigoplus_{s=-g}^g H(C_s,d_s),
\end{equation}
where $I^{\sharp}$ is the framed instanton Floer homology of a $3$-manifold, defined in \cite{kronheimer2010instanton} and $S^{3}_{2g+1}(K)$ is the $(2g+1)$-{\bf Dehn surgery} on the knot $K\subset S^3$.
\end{itemize}
\end{enumerate}






\section{Knot $7_6$}\label{sec3}
In this section, we prove the following theorem, and attempt to find the upper bound of $\dim I^{\sharp}(S^3_{2g+1}(K))$ where $ K = 7_6$. 

\begin{thm}
	Suppose $K=7_6$. Let $KHI(K)$ be the instanton knot homology of $K$ and
	$$d^i_j: KHI(K,i)\rightarrow KHI(K,j)$$
	are the differentials on $KHI(K)$ introduced in Theorem \ref{thm: Li-Ye}. Suppose further that for all $i,j\in\mathbb{Z}$ such that $|i-j|\geq 2$, we have $d^i_j=0$. Then 
	\[\dim I^{\sharp}(S^3_{5}(K))\leq 17.\]
\end{thm}


\bigskip

Since $K=7_6$ has genus $2$, from Theorem \ref{thm: KM} we have
$$ KHI(K) = KHI(K,2)\oplus KHI(K,1)\oplus KHI(K,0)\oplus KHI(K,-1)\oplus KHI(K,-2).$$
Furthermore, $K$ is an alternating knot, and the Alexander polynomial of the knot is
\[\Delta_{K}(t)=-t^2+5t-7+5t^{-1}-t^{-2},\]
so from Theorem \ref{thm: KM} we have
\begin{equation*}
    \begin{aligned}
        &\dim(KHI (K, 2))=\dim(KHI (K, -2))= 1,\\
        &\dim(KHI (K,1))=\dim(KHI (K,-1))= 5,\\
        &\dim(KHI (K, 0))= 7.
    \end{aligned}
\end{equation*}

From Theorem \ref{thm: Li-Ye}, we know
$$\dim H(C_2,d_2)=\dim H(C_{-2},d_{-2})=1.$$

Since the signature of the knot is $-2$, there are only two possibilities:
either
$$\dim (\ker d^1_0\slash \im d^2_1)=1$$
or
$$\dim (\ker d^{-1}_{-2}\slash\im d^0_1)=1$$ 
in $H(C_2,d_2)$ and all other summands of $H(C_2,d_2)$ are zero.





\begin{prop}
Suppose $K=7_6$ and
$$\dim (\ker d^1_0\slash \im d^2_1)=1.$$
Then
$$\dim I^{\sharp}(S^3_{5}(K))\leq 15.$$
\end{prop}

\textit{Proof.}
\textbf{Step 1: Computing $\dim \ker d^i_{i-1}$ and $\dim \im d^i_{i-1}$. }

From equation (\ref{eq: bent complex}), we know
$$\dim I^{\sharp}(S^3_{5}(K))= 2+\dim(H(C_1,d_1))+\dim(H(C_0,d_0))+\dim(H(C_{-1},d_{-1})).$$

From the second and third part of Theorem \ref{thm: Li-Ye},
\begin{equation}
    \begin{aligned}
        1=\dim H(C_2,d_2)=&\dim (\ker d^2_1)+ \dim (\ker d^1_0\slash \im d^2_1) + \dim(\ker d^0_{-1}/ \im d^1_0)\\
        &+\dim (\ker d^{-1}_{-2}/ \im d^0_{-1})+\dim (\mathbb{R}/\im d^{-1}_{-2}).
    \end{aligned}
\end{equation}

Since we have assumed that
\begin{equation}
    \dim (\ker d^1_0\slash \im d^2_1)=1,
\end{equation}
we can conclude that
\begin{equation}
    0= \dim (\ker d^2_1)= \dim(\ker d^0_{-1}/\im d^1_0)= \dim (\ker d^{-1}_{-2}/\im d^0_{-1})= \dim(\mathbb{R} / \im d^{-1}_{-2}).
\end{equation}
From the Fundamental Theorem of Linear Algebra, we have
$$\dim(\im d^2_1)=\dim KHI(K,2)-\dim (\ker d^2_1)=1.$$
Hence
$$\dim (\ker d^1_0) = 1+\dim(\im d^2_1)=2.$$
Again, from the Fundamental Theorem of Linear Algebra, we have
$$\dim(\im d^1_0)= \dim(KHI (K,1))-\dim (\ker d^1_0)=3.$$
From equation (5),
$$\dim(\ker d^0_{-1})-\dim (\im d^1_0)=0.$$
So
$$\dim(\ker d^0_{-1})=\dim (\im d^1_0)=3.$$
From the Fundamental Theorem of Linear Algebra, we have
$$\dim(\im d^0_{-1})=\dim(KHI(K,0))-\dim(\ker d^0_{-1})=4.$$
From equation (5),
$$\dim (\ker d^{-1}_{-2}) = \dim(\im d^0_{-1})=4.$$
From the Fundamental Theorem of Linear Algebra, we have
$$\dim(\im d^{-1}_{-2})=\dim(KHI(K,{-1}))-\dim (\ker d^{-1}_{-2})=1,$$
which is compatible with 
$$\dim(KHI(K,{-2})=\dim(\im d^{-1}_{-2}),$$
according to Theorem \ref{thm: Li-Ye}.

From symmetry, $\dim \ker d^i_j= \dim \ker d^{-i}_{-j}$
and $\dim \im d^i_j= \dim \im d^{-i}_{-j}.$

\textbf{Step 2 Estimating $\dim(H(C_1,d_1))$} 

From Theorem \ref{thm: Li-Ye},
\begin{equation}
    \begin{aligned}
       \dim(H(C_1,d_1))=&\dim(\ker~(d^1_2 \oplus d^1_0))+\dim(\ker~d^0_{-1}/\im(d^1_2 \oplus     d^1_0))\\+&\dim((\ker~d^{-1}_{-2}/\im~d^0_{-1}))+
        \dim(\mathbb{R}^3/\im d^{-1}_{-2}). 
    \end{aligned}
\end{equation}


$\ker(d^1_2 \oplus d^1_0)$ is $\ker d^{1}_{2} \cap \ker d^1_0 \subset KHI(K,1)\cong \mathbb{R}^5$, so from Theorem \ref{thm: A}, 

\begin{center}
${\dim(\ker d^{1}_{2}) + \dim(\ker d^1_0) - \dim (\mathbb{R}^5) }
\leq
{\dim(\ker d^{1}_{2} \cap \ker d^1_0)}
\leq
{\min (\dim(\ker d^{1}_{2}),\dim (\ker d^1_0))}$,
\end{center}


substituting numbers,
 
 \begin{center}
     $1\leq\dim(\ker (d^1_2 \oplus d^1_0))\leq2$.
 \end{center}

\bigskip

\textbf{Case 1 $\dim(\ker (d^1_2 \oplus d^1_0))=1$.}

From the Fundamental Theorem of Linear Algebra, we have
$$\dim(\im(d^1_2 \oplus d^1_0))= \dim \mathbb{R}^5-\dim(\ker (d^1_2 \oplus d^1_0))= 4,$$
so
\begin{equation*}
    \begin{aligned}
         \dim H(C_1,d_1) = 
         &\dim(\ker (d^1_2 \oplus d^1_0))
         +
         \dim(\ker d^0_{-1}/\im(d^1_2 \oplus d^1_0))\\
         &+
         \dim((\ker d^{-1}_{-2}/\im d^0_{-1})
         +
         \dim(\mathbb{R}/\im d^{-1}_{-2})=0.
    \end{aligned}
\end{equation*}
 
\textbf{Case 2 $\dim(\ker (d^1_2 \oplus d^1_0))=2$.}

From the Fundamental Theorem of Linear Algebra,
$$\dim(\im (d^1_2 \oplus d^1_0))= \dim \mathbb{R}^5-\dim(\ker (d^1_2 \oplus d^1_0))= 3,$$
so
\begin{equation*}
    \begin{aligned}
         \dim H(C_1,d_1) = 
         &\dim(\ker (d^1_2 \oplus d^1_0))
         +
         \dim(\ker d^0_{-1}/\im(d^1_2 \oplus d^1_0))\\
         &+
         \dim((\ker d^{-1}_{-2}/\im d^0_{-1})
         +
         \dim(\mathbb{R}/\im d^{-1}_{-2})=2.
    \end{aligned}
\end{equation*}

\bigskip
Therefore,
$\dim H(C_1,d_1) \leq 2$.

From Theorem \ref{thm: KM}, 
$\max\{\dim H(C_1,d_1)\}\cong\max\{\dim H(C_{-1},d_{-1}\}\cong \mathbb{R}^2$.


\bigskip
\textbf{Step 3: Estimating $H(C_0,d_0)$} 

From Theorem \ref{thm: Li-Ye},
$$\dim H(C_0,d_0)= 
\dim(\ker (d^0_1\oplus d^0_{-1}))
+
\dim((\ker (d^1_2\oplus d^{-1}_{-2})/\im(d^0_1\oplus d^0_{-1}))
+
\dim \mathbb{R}^2
+
\dim (\im (d^1_2\oplus d^{-1}_{-2}))
,$$
where
$\ker (d^1_2\oplus d^{-1}_{-2})$ is $\ker d^{-1}_{-2} \oplus \ker d^{1}_{2}$, and $\im (d^1_2\oplus d^{-1}_{-2})$ is $\im d^{-1}_{-2} \oplus \im d^{1}_{2}$.
Therefore,
$$\dim(\ker (d^1_2\oplus d^{-1}_{-2})) = \dim (\ker d^{-1}_{-2}) + \dim (\ker d^{1}_{2}) = 8,$$
$$\dim(\im (d^1_2\oplus d^{-1}_{-2})) = \dim (\im d^{-1}_{-2}) + \dim (\im d^{1}_{2}) = 2.$$

$\dim(\ker (d^0_1\oplus d^0_{-1}))$ is $\ker d^{1}_{2} \cap \ker d^1_0 \subset KHI(K,0)\cong \mathbb{R}^7$, so from Theorem \ref{thm: A},
$$\dim (\ker d^0_{1})
+ 
\dim (\ker d^0_{-1} )
- 
\dim (\mathbb{R}^7)
\leq
\dim (\ker (d^{1}_{2} \cap \ker d^1_0))
\leq
\min (\dim (\ker d^0_{1}),\dim (\ker d^0_{-1})).
$$
Substituting numbers,
$$
0
\leq 
{\dim (\ker d^0_{1} \cap \ker d^0_{-1}) }
\leq 
3
$$

\textbf{Case 1 $\dim(\ker (d^0_1\oplus d^0_{-1}))= 0$.}

From the Fundamental Theorem of Linear Algebra,
$$\dim(\im (d^0_1\oplus d^0_{-1}))= \dim \mathbb{R}^7-\dim(\ker (d^0_1\oplus d^0_{-1}))=7,$$
so
\begin{equation*}
    \begin{aligned}
       \dim H(C_0,d_0)= 
        &
        \dim(\ker (d^0_1\oplus d^0_{-1}))
        +
        \dim((\ker (d^1_2\oplus d^{-1}_{-2})/\im(d^0_1\oplus d^0_{-1}))\\
        &
        +\dim \mathbb{R}^2/\dim (\im (d^1_2 \oplus d^{-1}_{-2}))
        =3.
    \end{aligned}
\end{equation*}


\textbf{Case 2 $\dim(\ker (d^0_1\oplus d^0_{-1}))= 1$.}

From the Fundamental Theorem of Linear Algebra,
$$\dim(\im (d^0_1\oplus d^0_{-1}))= \dim \mathbb{R}^7-\dim(\ker (d^0_1\oplus d^0_{-1}))=6,$$
so
\begin{equation*}
    \begin{aligned}
       \dim H(C_0,d_0)= 
        &\dim(\ker (d^0_1\oplus d^0_{-1}))
        +
        \dim((\ker (d^1_2\oplus d^{-1}_{-2})/\im(d^0_1\oplus d^0_{-1}))\\
        &+
        \dim \mathbb{R}^2/\dim (\im (d^1_2 \oplus d^{-1}_{-2}))
        =5.
    \end{aligned}
\end{equation*}

\textbf{Case 3 $\dim(\ker (d^0_1\oplus d^0_{-1}))= 2$.}

From the Fundamental Theorem of Linear Algebra,
$$\dim(\im (d^0_1\oplus d^0_{-1}))= \dim \mathbb{R}^7-\dim(\ker (d^0_1\oplus d^0_{-1}))=5,$$
so
\begin{equation*}
    \begin{aligned}
       \dim H(C_0,d_0)= 
        &\dim(\ker (d^0_1\oplus d^0_{-1}))
        +
        \dim((\ker (d^1_2\oplus d^{-1}_{-2})/\im(d^0_1\oplus d^0_{-1}))\\
        &+
        \dim \mathbb{R}^2/\dim (\im (d^1_2 \oplus d^{-1}_{-2}))
        =7.
    \end{aligned}
\end{equation*}

\textbf{Case 4 $\dim(\ker (d^0_1\oplus d^0_{-1}))= 3$.}

From the Fundamental Theorem of Linear Algebra,
$$\dim(\im (d^0_1\oplus d^0_{-1}))= \dim \mathbb{R}^7-\dim(\ker (d^0_1\oplus d^0_{-1}))=4,$$
so
\begin{equation*}
    \begin{aligned}
       \dim H(C_0,d_0)= 
        &\dim(\ker (d^0_1\oplus d^0_{-1}))
        +
        \dim((\ker (d^1_2\oplus d^{-1}_{-2})/\im(d^0_1\oplus d^0_{-1}))\\
        &+
        \dim \mathbb{R}^2/\dim (\im (d^1_2 \oplus d^{-1}_{-2}))
        =9.
    \end{aligned}
\end{equation*}
Therefore,
$\dim H(C_0,d_0) \leq 9$.

\bigskip
In conclusion, When
$\dim (\ker d^1_0\slash \im d^2_1)=1$, $K=7_6$,
$$ \dim I^{\sharp}(S^3_{5}(K))\leq 15.$$

\begin{prop}
Suppose $K=7_6$ and
$$\dim (\ker d^{-1}_{-2}\slash \im d^0——{-1})=1.$$
Then
$$\dim I^{\sharp}(S^3_{5}(K))\leq 17.$$
\end{prop}

\textit{Proof.}
\textbf{Step 1: Computing $\dim \ker d^i_{i-1}$ and $\dim \im d^i_{i-1}$. }

From (\ref{eq: bent complex}), we know
$$\dim I^{\sharp}(S^3_{5}(K))= 2+\dim(H(C_1,d_1))+\dim(H(C_0,d_0))+\dim(H(C_{-1},d_{-1}))$$

From the second and third part of Theorem \ref{thm: Li-Ye},
\begin{equation}
    \begin{aligned}
        1=\dim H(C_2,d_2)=&\dim (\ker d^2_1)+ \dim (\ker d^1_0\slash \im d^2_1) + \dim(\ker d^0_{-1}/ \im d^1_0)\\
        &+\dim (\ker d^{-1}_{-2}/ \im d^0_{-1})+\dim (\mathbb{R}/\im d^{-1}_{-2})
    \end{aligned}
\end{equation}

Since we have assumed that
\begin{equation}
    \dim (\ker d^1_0\slash \im d^2_1)=1,
\end{equation}
we can conclude that
\begin{equation}
    0= \mathbb{R}/\im d^{-1}_{-2}= \dim(\ker d^2_1)= \dim (\ker d^1_0/ \im d^2_1)= \dim(\ker d^0_{-1}/ \im d^1_0)
\end{equation}

From the Fundamental Theorem of Linear Algebra, we have
$$\dim(\ker d^{-1}_{-2})=\dim KHI(K,-2)-\dim (\im d^{-1}_{-2})=4.$$
From equation (9),
$$\dim (\im d^0_{-1}) = \dim(\ker d^{-1}_{-2})-1=3 .$$
Again, from the Fundamental Theorem of Linear Algebra, we have
$$\dim(\ker d^0_{-1})= \dim(KHI (K,0))-\dim (\im d^0_{-1})=4.$$
From equation (9),
$$\dim(\ker d^0_{-1})=\dim (\im d^1_0)=4.$$
From the Fundamental Theorem of Linear Algebra, we have
$$\dim(\ker d^1_0)=\dim(KHI(K,1))-\dim(\im d^1_0\ker d^0_{-1})=1$$
From equation (9),
$$\dim (\im d^2_1) = \dim(\ker d^1_0)= 1.$$
Again, from the Fundamental Theorem of Linear Algebra, we have
$$\dim(\ker d^2_1)=\dim(KHI(K,2))-\dim (\im d^2_1)=0.$$

From symmetry, $\dim \ker d^i_j= \dim \ker d^{-i}_{-j}$
and $\dim \im d^i_j= \dim \im d^{-i}_{-j}.$

\textbf{Step 2 Estimating $\dim(H(C_1,d_1))$} 

From Theorem \ref{thm: Li-Ye},
\begin{equation}
    \begin{aligned}
       \dim(H(C_1,d_1))=&\dim(\ker~(d^1_2 \oplus d^1_0))+\dim(\ker~d^0_{-1}/\im(d^1_2 \oplus     d^1_0))\\+&\dim((\ker~d^{-1}_{-2}/\im~d^0_{-1}))+
        \dim(\mathbb{R}^3/\im d^{-1}_{-2}). 
    \end{aligned}
\end{equation}


$\ker(d^1_2 \oplus d^1_0)$ is $\ker d^{1}_{2} \cap \ker d^1_0 \subset KHI(K,1)\cong \mathbb{R}^5$, so from Theorem \ref{thm: A}, 

\begin{center}
${\dim(\ker d^{1}_{2}) + \dim(\ker d^1_0) - \dim (\mathbb{R}^5) }
\leq
{\dim(\ker d^{1}_{2} \cap \ker d^1_0)}
\leq
{min (\dim(\ker d^{1}_{2}),\dim (\ker d^1_0))}$,
\end{center}

substituting numbers,
 
 \begin{center}
     $0\leq\dim(\ker (d^1_2 \oplus d^1_0))\leq1$.
 \end{center}

\bigskip

\textbf{Case 1 $\dim(\ker (d^1_2 \oplus d^1_0))=0$.}

From the Fundamental Theorem of Linear Algebra,
$$\dim(\im(d^1_2 \oplus d^1_0))= \dim \mathbb{R}^5-\dim(\ker (d^1_2 \oplus d^1_0))= 5$$
so,
\begin{equation*}
    \begin{aligned}
         \dim H(C_1,d_1) = 
         &\dim(\ker (d^1_2 \oplus d^1_0))
         +
         \dim(\ker d^0_{-1}/\im(d^1_2 \oplus d^1_0))\\
         &+
         \dim((\ker d^{-1}_{-2}/\im d^0_{-1})
         +
         \dim(\mathbb{R}/\im d^{-1}_{-2})=0.
    \end{aligned}
\end{equation*}

\textbf{Case 2 $\dim(\ker (d^1_2 \oplus d^1_0))=1$.}

From the Fundamental Theorem of Linear Algebra,
$$\dim(\im (d^1_2 \oplus d^1_0))= \dim \mathbb{R}^5-\dim(\ker (d^1_2 \oplus d^1_0))= 4$$
so,
\begin{equation*}
    \begin{aligned}
         \dim H(C_1,d_1) = 
         &\dim(\ker (d^1_2 \oplus d^1_0))
         +
         \dim(\ker d^0_{-1}/\im(d^1_2 \oplus d^1_0))\\
         &+
         \dim((\ker d^{-1}_{-2}/\im d^0_{-1})
         +
         \dim(\mathbb{R}/\im d^{-1}_{-2})=2.
    \end{aligned}
\end{equation*}

\bigskip
Therefore,
$\dim H(C_1,d_1) \leq 2$.

From Theorem \ref{thm: KM}, 
$\dim H(C_1,d_1) \cong \dim H(C_{-1},d_{-1}) \cong \mathbb{R}^2$, so
 $\dim H(C_{-1},d_{-1}) \leq2$.


\bigskip
\textbf{Step 3: Estimating $H(C_0,d_0)$} 

From Theorem \ref{thm: Li-Ye},
$$\dim H(C_0,d_0)= 
\dim(\ker (d^0_1\oplus d^0_{-1}))
+
\dim((\ker (d^1_2\oplus d^{-1}_{-2})/\im(d^0_1\oplus d^0_{-1}))
+
\dim \mathbb{R}^2
+
\dim (\im (d^1_2\oplus d^{-1}_{-2}))
,$$
where
$\ker (d^1_2\oplus d^{-1}_{-2})$ is $\ker d^{-1}_{-2} \oplus \ker d^{1}_{2}$, and $\im (d^1_2\oplus d^{-1}_{-2})$ is $\im d^{-1}_{-2} \oplus \im d^{1}_{2}$.
Therefore,
$$\dim(\ker (d^1_2\oplus d^{-1}_{-2})) = \dim (\ker d^{-1}_{-2}) + \dim (\ker d^{1}_{2}) = 8$$
$$\dim(\im (d^1_2\oplus d^{-1}_{-2})) = \dim (\im d^{-1}_{-2}) + \dim (\im d^{1}_{2}) = 2.$$

$\dim(\ker (d^0_1\oplus d^0_{-1}))$ is $\ker d^{1}_{2} \cap \ker d^1_0 \subset KHI(K,0)\cong \mathbb{R}^7$, so from Theorem \ref{thm: A}
$$\dim (\ker d^0_{1})
+ 
\dim (\ker d^0_{-1} )
- 
\dim (\mathbb{R}^7)
\leq
\dim (\ker (d^{1}_{2} \cap \ker d^1_0))
\leq
\min (\dim (\ker d^0_{1}),\dim (\ker d^0_{-1})).
$$
Substituting numbers,
$$1\leq {\dim (\ker d^0_{1} \cap \ker d^0_{-1}) }\leq 4.$$

\textbf{Case 1 $\dim(\ker~(d^0_1\oplus d^0_{-1}))= 1$.}

From the Fundamental Theorem of Linear Algebra,
$$\dim(\im~(d^0_1\oplus d^0_{-1}))= \dim \mathbb{R}^7-\dim(\ker~(d^0_1\oplus d^0_{-1}))=6,$$
so
\begin{equation*}
    \begin{aligned}
       \dim H(C_0,d_0)= 
        &\dim(\ker (d^0_1\oplus d^0_{-1}))
        +
        \dim((\ker (d^1_2\oplus d^{-1}_{-2})/\im(d^0_1\oplus d^0_{-1}))\\
        &+
        \dim \mathbb{R}^2/\dim (\im (d^1_2 \oplus d^{-1}_{-2}))
        =3.
    \end{aligned}
\end{equation*}


\textbf{Case 2 $\dim(\ker (d^0_1\oplus d^0_{-1}))= 2$.}

From the Fundamental Theorem of Linear Algebra,
$$\dim(\im (d^0_1\oplus d^0_{-1}))= \dim \mathbb{R}^7-\dim(\ker (d^0_1\oplus d^0_{-1}))=5,$$
so
\begin{equation*}
    \begin{aligned}
       \dim H(C_0,d_0)= 
        &\dim(\ker (d^0_1\oplus d^0_{-1}))
        +
        \dim((\ker (d^1_2\oplus d^{-1}_{-2})/\im(d^0_1\oplus d^0_{-1}))\\
        &+
        \dim \mathbb{R}^2/\dim (\im (d^1_2 \oplus d^{-1}_{-2}))
        =5.
    \end{aligned}
\end{equation*}

\textbf{Case 3 $\dim(\ker (d^0_1\oplus d^0_{-1}))= 3$.}

From the Fundamental Theorem of Linear Algebra,
$$\dim(\im (d^0_1\oplus d^0_{-1}))= \dim \mathbb{R}^7-\dim(\ker (d^0_1\oplus d^0_{-1}))=4,$$
so
\begin{equation*}
    \begin{aligned}
       \dim H(C_0,d_0)= 
        &\dim(\ker (d^0_1\oplus d^0_{-1}))
        +
        \dim((\ker (d^1_2\oplus d^{-1}_{-2})/\im(d^0_1\oplus d^0_{-1}))\\
        &+
        \dim \mathbb{R}^2/\dim (\im (d^1_2 \oplus d^{-1}_{-2}))
        =7.
    \end{aligned}
\end{equation*}

\textbf{Case 4 $\dim(\ker (d^0_1\oplus d^0_{-1}))= 4$.}

From the Fundamental Theorem of Linear Algebra,
$$\dim(\im (d^0_1\oplus d^0_{-1}))= \dim \mathbb{R}^7-\dim(\ker (d^0_1\oplus d^0_{-1}))=3,$$
so
\begin{equation*}
    \begin{aligned}
       \dim H(C_0,d_0)= 
        &\dim(\ker (d^0_1\oplus d^0_{-1}))
        +
        \dim((\ker (d^1_2\oplus d^{-1}_{-2})/\im(d^0_1\oplus d^0_{-1}))\\
        &+
        \dim \mathbb{R}^2/\dim (\im (d^1_2 \oplus d^{-1}_{-2}))
        =9.
    \end{aligned}
\end{equation*}
Therefore,
$\dim H(C_0,d_0)\leq 9$.

When
$\dim (\ker d^{-1}_{-2}\slash \im d^0_{-1})=1$, $K=7_6$,
$$\dim I^{\sharp}(S^3_{5}(K))\leq 17.$$

\bigskip
In conclusion, when $K=7_6$ 
$$\dim I^{\sharp}(S^3_{5}(K))\leq17.$$

 

\bibliographystyle{alpha}
\bibliography{ref.bib}



\end{document}
