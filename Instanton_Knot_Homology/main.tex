\documentclass{amsart}
\usepackage{amsmath}
\usepackage{geometry}
\usepackage{amsthm}
\usepackage{fancyhdr}


\title{The calculation of dimensions of Framed instanton Floer homology of $3$-dimension flow with Knot K from Ball $3$D after Dehn Surgery}
\author{Yu-Chia Jen,Haoze Pang}
\address{International Division, Guangzhou Zhixin High School}
\curraddr{12 Shuiyin Erheng Road, Yuexiu District, Guangzhou, Guangdong, China}
\email{692578536@qq.com}
\newtheorem{thm}{Theorem}[section]
\theoremstyle{definition}
\newtheorem{defn}[thm]{Definition}
%\newtheorem{thm}{Theorem}[section]
\newtheorem{cor}[thm]{Corollary}
%\theoremstyle{definition}
%\newtheorem{defn}[thm]{Definition}
\newcommand{\im}{{\rm im}}
\date{November 2021}


\begin{document}
\begin{abstract}
\end{abstract}
\maketitle
\tableofcontents


\section{Introduction}
In this paper, we will estimate the possible maximum dimension of homology of chain complex.

Different ways of glueing back $N(K) give rise to different spaces, which depends only on the image of mk. S


\section{Preliminaries}
\begin{defn}
Knot
\end{defn}
A knot $K$ is an embedded circle inside the $3$-dimensional Euclidean space $\mathbb{R}^3$.
\[K:S^1\hookrightarrow\mathbb{R}^3\]
Knots are defined inside the $3$-dimensional space, while it is hard  to describe and study objects in dimension $3$. So we introduce the following definition for our convenience.

\begin{defn}
Knot diagram.
\end{defn}
A knot diagram is a projection of the knot onto some plane in $\mathbb{R}^3$ which only allow double intersection. There are infinitely many different embeddings of circles into $\mathbb{R}^3$, but many of them can have identical properties. So in order to study different kinds of knots, we make the following definition.

\begin{defn}
Alternating Knot
\end{defn}
For some knot diagrams, if we transverse the knot, then over and under crossing appears alternatively.


\begin{defn}
Kernel of $f$
\end{defn}
Suppose $V$ and $W$ are two vectors spaces,$f:V\rightarrow W$ is a linear map.In linear algebra, the kernel of a linear map $f:V\rightarrow W$ is the set of $v$ of $V$ for which $f(v)=0$, $0$ represents the zero vector in $W$:
\[\ker(f)= \{v\in V\mid f(v)=\vec{0} \in W\}\]

\begin{defn}
Image of $f$
\end{defn}
Suppose $V$ and $W$ are two vectors spaces,$f:V\rightarrow W$ is a linear map.In linear algebra, the image of a linear map $f:V\rightarrow W$ is the set of $w$ of $W$ for which satisfies $f(v)=w,\exists v \in V$:
\[\im (f)= \{w\in W \mid\exists v\in V,f(v)= w\}\]

\begin{thm}
Fundamental theorem of linear algebra
\end{thm}
Suppose $f:U\rightarrow V$ is a linear map between finite dimensional vector spaces,then:
\[\dim \ker(f)+\dim \im(f)=\dim U\]

\begin{thm}
Supposed $A, B$ are both subspace of $C$,then we have:
\[max\{0,\dim A+\dim B-\dim C\}\leq \dim(A\cap B)\leq min\{\dim A, \dim B\}\]
\end{thm}

\begin{thm}
Alexander polynomial
\end{thm}
For any knot $K\subset \mathbb{R}^3$, there exists a polynomial that is associated to it.This polynomial is called Alexander polynomial.The definition of Alexander polynomial is in(ADD REFERENCE):\\\\
(1,1)\;\;\;\;\;\;\;\;\;\;\;\;\;\;$\Delta_K(t)=a_{n}t^{n}+a_{n-1}t^{n-1}+...+a_{0}+...+a_{-(n-1)}t^{-(n-1)}+a_{-n}t^{-n}.$\\\\
Alexander polynomial satisfies the following properties:
\begin{enumerate}
\item
$\Delta_K(1)=1.$
\item
For any $i\in\mathbb{Z}$,we have $a_{i}=a_{-i}.$
\end{enumerate}

\begin{thm}
For any knot $K \subset\mathbb{R}^3$, there is a vector space $KHI(K)$ associated to $K$. It is called the instanton knot homology of $K$(ADD REFERENCE). It satisfies the following properties:
\begin{enumerate}
\item
There is a decomposition:
$KHI(K)=\oplus KHI(K,i)$.
Here $\oplus$ means taking direct sum. The vector space $KHI(K,i)$ is called the $i-th$ graded part of $KHI(K)$.
\item
Suppose $\Delta_K(t)$ is the Alexander polynomial of $K$ and $a_{i}'s$ are the coefficients of $\Delta_K$ as in (1,1). Then we have
\[\dim KHI(K,i)\geq |a_{i}|\]
for all $i\in\mathbb{Z}$ and the integer ($\dim KHI(K,i)-|a_{i}|$) is divisible by 2 for all $i\in\mathbb{Z}$. Here dim means the dimension of a vector space.
\item
For any $i\in\mathbb{Z}$, we have
\[KHI(K,i)\cong KHI(K,-i)\]
\item
If $K$ is an alternating knot, then we have
\[\dim KHI(K,i)=|a_{i}|\]
for all $i\in\mathbb{Z}$.
\end{enumerate}
\end{thm}

\begin{defn}
Chain complex and Differential
\end{defn}
Suppose we have a sequence of vectors spaces $C_{1},C_{2}$,... Suppose for each $i\in\mathbb{Z}$,we have a linear map
\[d_{i}:C_{i}\rightarrow C_{i-1}.\]
$d$ is called the differential.The couple $({C_{i},{d_{i}}})$ is called a chain complex if the composition of maps
\[d_{i}\circ d_{i+1}:C_{i}\rightarrow C_{i-1}\]
is always zero for all $i\in\mathbb{Z}$.Let
\[C=\oplus C_{i}\]
and let d be the linear map
\[d:C\rightarrow C\]
so that d restricts to $d_{i}$ on $C_{i}$, i.e.,
\[d|_{C_{i}}=d_{i}:C_{i}\rightarrow C_{i-1}\]
Then we usually write$(C,d)$ instead of $({C_{i}},{d_{i}})$

\begin{defn}
Homology
\end{defn}
Suppose $(C,d)$ is a chain complex. Define
\[(H,d)=ker (d)/\im (d)\]
to be homology of $(C,d)$. Here $\ker$($d$) means the kernel of $d$ and $\im$($d$) means the image of $d$.
The symbol/means the quotient of vectors spaces. If
\[C=\oplus C_{i}\]
and let $d$ be the linear map
\[d:C\rightarrow C\]
so that $d$ restricts to $d_{i}$ on $C_{i}$, i.e.,
\[d|_{C_{i}}=d_{i}:C_{i}\rightarrow C_{i-1}\]
Then we can define
\[H_{i}(C,d)=\ker(d_{i})/\im(d_{i+1})\]
and we have:
\[H(C,d)=\oplus H_{i}(C,d).\]

\begin{thm}
(Proposition of Li-Ye)
\end{thm}
In(ADD REFERENCE) , two researchers proved the following:
\begin{enumerate}
\item
For any $i,j\in \{-g,...,g\}$ ($g=g(k)$, knot genus),there exist a map:
\[d^{i}_{j}:KHI(k,i)\rightarrow KHI(k,j)\]
Such that: if $k+j=2i$ then $d^{i}_{j}\circ d^{k}_{i}=0$
\item
For any knot $K \subset\mathbb{R}^3$, the dimension of homology $(C_{2},d_{2})$ and $(C_{-2},d_{-2})$ is always 1.
\end{enumerate}


\section{Knot $7_6$}
\textbf{Basic Information:}

Alexander Polynomial: 	$1-5t+7t^2-5t^3+t^4$ 

Genus : $2$

Signature: $-2$
\bigskip

\textbf{Deductions:} 
\bigskip

From Proposition(Li-Ye): 

\begin{center}
  $ \dim H(C_2,d_2)=\dim H(C_{-2},d_{-2})=1$
\end{center}
   


Since the signature of the knot is $-2$, we know that 
\begin{center}
$\dim (\ker d^1_0\slash \im d^2_1)=1$, or
$\dim (\ker d^{-1}_{-2}\slash\im d^0_1)=1$
\end{center}
in $H(C_2,d_2)$, and the rest of its component is $0$.



%case 1
\bigskip
\textbf{Case 1:} $\dim (\ker d^1_0\slash \im d^2_1) =1$

%from xx, we know that
$\dim H(C_2,d_2)= \dim(\ker d^2_1)+\dim (\ker d^1_0\slash \im d^2_1)$



Under such circumstance, we found:

\begin{center}
$\dim (\ker d^2_1) = 1$, and $\dim(\im d^2_1)=1$
 
$\dim (\ker d^1_0) = 2$, and $\dim(\im d^1_0) = 3$
 
$\dim (\ker d^0_{-1}) = 3$,  and $\dim(\im d^0_{-1})= 4$
 
$\dim (\ker d^{-1}_{-2}) = 4$, and $dim(\im d^{-1}_{-2}) = 1$
\end{center}

 


















 %hc1d1
 \bigskip
 Now, we calculate $H(C_1,d_1)$.


$\ker (d^1_2 \oplus d^1_0)$ is $\ker d^{1}_{2} \cap \ker d^1_0$ in $ KHI(\Delta k,1)$( $\mathbb{R}^5$) 

So,
\begin{center}
${\dim(\ker d^{1}_{2}) + \dim(\ker d^1_0) - \dim (\mathbb{R}^5) }
\leq
{\dim(\ker d^{1}_{2} \cap \ker d^1_0)}
\leq
{\min (\dim(\ker~d^{1}_{2}),\dim (\ker~d^1_0))}$
\end{center}


Substituting numbers, we get: 
 
 \begin{center}
     $1\leq\dim(\ker (d^1_2 \oplus d^1_0))\leq2$
 \end{center}
 

\bigskip
From the fundamental theorem of linear algebra,

\begin{center}
    $\dim(\ker~(d^1_2 \oplus d^1_0))+\dim(\im~(d^1_2 \oplus d^1_0))= \dim \mathbb{R}^5=5$
\end{center}





%sit1
\bigskip

\textbf{Situation 1}

\begin{center}
$\dim(\ker(d^1_2 \oplus d^1_0))=1$
            
$\dim(\im(d^1_2 \oplus d^1_0))=4$
\end{center}

            
\bigskip
In this situation,

\begin{center}
    $\dim H(C_1,d_1) = 
\dim(\ker~(d^1_2 \oplus d^1_0))
+
\dim(\ker~d^0_{-1}/\im(d^1_2 \oplus d^1_0))
+
\dim((\ker~d^{-1}_{-2}/im~d^0_{-1})
+
\dim(\mathbb{R}^3/\im d^{-1}_{-2})=0$
\end{center}


%sit2
\bigskip
\textbf{Situation 2}

$\dim(\ker~(d^1_2 \oplus d^1_0))=2$
            
$dim(\im~(d^1_2 \oplus d^1_0))=3$
            
\bigskip
In this situation,

\bigskip
$\dim H(C_1,d_1) 
= 
\dim(\ker~(d^1_2 \oplus d^1_0))
+
\dim(\ker~d^0_{-1}/\im(d^1_2 \oplus d^1_0))
+
\dim((\ker~d^{-1}_{-2}/im~d^0_{-1})
+
\dim(R^3/\im d^{-1}_{-2})
=
2$

\bigskip
In case 1,
The upper bound of $\dim H(C_1,d_1)$ is $2$ .

From symmetry, we know that the upper bound of $\dim H(C_{-1},d_{-1})$ is also $2$.






%h00
\bigskip
Now, we calculate $H(C_0,d_0)$.

$\dim(\ker (d^1_2\oplus d^{-1}_{-2})) = \dim (\ker d^{-1}_{-2}) + \dim (\ker d^{1}_{2}) = 8$

$\dim(\im (d^1_2\oplus d^{-1}_{-2})) = \dim (\im d^{-1}_{-2}) + \dim (\im d^{1}_{2}) = 2$

%situs
\bigskip
$\dim (\ker d^0_{1})
+ 
\dim~(\ker d^0_{-1} )
- 
\dim R^7
\leq
\dim (\ker (d^{1}_{2} \cap \ker~d^1_0))
\leq
min (\dim (\ker d^0_{1}),\dim (\ker d^0_{-1}))$

So,

$0\leq {dim (ker d^0_{1} \cap ker d^0_{-1}) }\leq 3$

$\dim(\ker~(d^0_1\oplus d^0_{-1}))
= 
\dim ((\ker d^0_{1} \cap \ker d^0_{-1})) $

\bigskip
From the fundamental theorem of linear algebra,

$\dim(\ker~(d^0_1\oplus d^0_{-1}))
+
\dim(\im~(d^0_1\oplus d^0_{-1}))
= 
dim R^7
=7$



\bigskip
\textbf{Situation 1}

$\dim(\ker~(d^0_1\oplus d^0_{-1}))= 0$

$\dim(\im~(d^0_1\oplus d^0_{-1}))=7$

\bigskip
So,

$\dim~H(C_0,d_0)= 
\dim(\ker~(d^0_1\oplus d^0_{-1}))
+
\dim((\ker~(d^1_2\oplus d^{-1}_{-2})/\im(d^0_1\oplus d^0_{-1}))
+
\dim R^2/+
\dim (\im~(d^1_2\oplus d^{-1}_{-2}))
=1$



\bigskip
\textbf{Situation 2}

$\dim(\ker~(d^0_1\oplus d^0_{-1}))= 1$

$\dim(\im~(d^0_1\oplus d^0_{-1}))=6$

\bigskip
So,

$\dim~H(C_0,d_0)= 
\dim(\ker~(d^0_1\oplus d^0_{-1}))
+
\dim((\ker~(d^1_2\oplus d^{-1}_{-2})/\im(d^0_1\oplus d^0_{-1}))
+
\dim R^2/+
\dim (\im~(d^1_2\oplus d^{-1}_{-2}))
=3$


\bigskip
\textbf{Situation 3}

$\dim(\ker~(d^0_1\oplus d^0_{-1}))= 2$

$\dim(\im~(d^0_1\oplus d^0_{-1}))=5$

\bigskip
So,

$\dim~H(C_0,d_0)= 
\dim(\ker~(d^0_1\oplus d^0_{-1}))
+
\dim((\ker~(d^1_2\oplus d^{-1}_{-2})/\im(d^0_1\oplus d^0_{-1}))
+
\dim R^2/+
\dim (\im~(d^1_2\oplus d^{-1}_{-2}))
=5$

\bigskip
\textbf{Situation 4}

$\dim(\ker~(d^0_1\oplus d^0_{-1}))= 3$

$\dim(\im~(d^0_1\oplus d^0_{-1}))=4$

\bigskip
So,

$\dim~H(C_0,d_0)= 
\dim(\ker~(d^0_1\oplus d^0_{-1}))
+
\dim((\ker~(d^1_2\oplus d^{-1}_{-2})/\im(d^0_1\oplus d^0_{-1}))
+
\dim R^2/+
\dim (\im~(d^1_2\oplus d^{-1}_{-2}))
=7$


\bigskip
In case 1, the upper bound of $\dim H(C_0,d_0)$ is $7$.

\bigskip
In case 1, 
max{
$\dim H(C_2,d_2)
+
\dim H(C_1,d_1)
+
\dim H(C_0,d_0)
+
\dim H(C_{-1},d_{-1})
+
\dim H(C_{-2},d_{-2})$}
$=$
$15$





  
  
  



%case2
\bigskip

 
 
 
 
 
 
 
 
 




\end{document}
