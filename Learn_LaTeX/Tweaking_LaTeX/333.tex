\documentclass[a4paper,12pt]{article}
\usepackage{geometry}
\usepackage{fancyhdr}
%\renewcommand{\familydefault}{\sfdefault}
%\usepackage{fontenc}
%\usepackage{helvet}
%above 3 lines are the commands that implement Helvetica font
\usepackage{fontspec}
\setmainfont{Arial}
\usepackage{parskip}
\usepackage{microtype}
\usepackage{blindtext}
\usepackage[british]{babel}
\usepackage{blindtext}
\usepackage{graphicx}
\usepackage{placeins}
%line above prevents floating over sections
\usepackage{array}
\usepackage[font=footnotesize]{caption}
\usepackage{booktabs}
\setlength{\heavyrulewidth}{1.5pt}
\usepackage{siunitx}
\usepackage{url}
\begin{document}
{\pagestyle{empty}
	\begin{center}
		{\LARGE\bfseries Title first word starts with a capital\\
		
		}
		\vspace{\baselineskip}
		{\itshape Haoze Pang \\
			11019521\\
		}
		\vspace{\baselineskip}	
		Department of Physics and Astronomy\\
		The University of Manchester\\
		\vspace{\baselineskip}	
		First Year Laboratory Report\\
		\vspace{\baselineskip}	
		{\today}\\
		\vspace{2\baselineskip}		
		This experiment was performed in collaboration with \textit{Alec Bury 10833524}.
	\end{center}
	\vspace{4\baselineskip}
	\textbf{Abstract}
	
	Insert your abstract here.  Remember to delete the original text.  The abstract should be about 50 ‒ 200 words, and give a very brief overview of the experiment and main results.  Abstracts are important [1] because they set the tone for the rest of the document.  If your abstract includes a reference citation then give the source below the abstract.  This ensures that the abstract is self-contained.  Do not change the fonts or format of this Title Page, which is not numbered.  The rest of your report should start on the next page, page 2.
\clearpage}
\section{Introduction}
\bfseries
ffffff
\section{Theory}
\section{Experimental Method}
\section{Data and Analysis}
\section{Results and Discussion}
\section{Summary}
\end{document}